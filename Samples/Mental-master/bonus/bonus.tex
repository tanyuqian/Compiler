\documentclass[11pt,a4paper]{article}
\usepackage{ctex}
\usepackage[margin=1in]{geometry}

\title{Mental特性和优化汇总}
\author{柯嵩宇}
\begin{document}

\maketitle
\tableofcontents

\newpage
\section{visitor模式:from AST to linear IR}
	以visitor模式完全手写了AST到IR的visitor。(ANTLR生成CST,用ANTLR的CST生成我的AST,然后用我写的visitor生成IR)
\section{逻辑表达式短路求值}
\subsection{naive的短路求值}
	对于二元运算的逻辑与或,可以先计算左边的元素,如果左侧表达式的结果可以决定整个表达式的值,那么就保存结果同时跳过右侧表达式的计算。
\subsection{修改逻辑与或运算的结合性}
	在superloop中,出现了一个巨大的逻辑与运算,如果按照上面的方法短路求值,那么在执行过程中会出现大量的跳转。一个比较机智的优化就是修改逻辑与或运算的结合性。如果确定表达式结果的子表达式出现在整个表达式的前面,那么就可以用比较少的跳转离开这个表达式的计算。
\subsection{Super Expression}
	修改逻辑与或运算结合性的优化并不是那么完美,有一个更厉害的优化:定义广义逻辑与或运算,允许逻辑与或运算有2个以上的操作数,这样在处理的时候就可以看到所有的操作数,然后就可以用1次的跳转离开表达式的求值。
\section{Linear Scan的局部寄存器分配}
	线性扫描每一个Basic Block,然后以贪心的方式分配寄存器
\section{伪活性分析:数据引用计数}
	对于非变量的数据计算引用次数,如果引用次数到0,就不需要在保存到寄存器中。
\section{超越语义的print语句}
	在Mx语言中,print和println的函数只能是一个字符串,对于输出语句中的字符串加法(作为参数传给print函数)而言,这个加法的结果在以后完全不会被用到,所以在这个字符串加法其实是没有必要进行的,可以通过拆分print(a+b)为print(a)和print(b)来优化输出语句的效率。当然由于MIPS本身存在输出整数的syscall,所以如果要完全优化的话,可以把print(toString(x))直接优化掉(我并没有写到这一步,我就写了加法的拆分)。


\end{document}
